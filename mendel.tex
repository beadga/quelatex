\documentclass{report}

\usepackage{latexsym, amsfonts, amsmath}
\usepackage[utf8]{inputenc}
\usepackage{graphicx}
\usepackage[spanish, mexico]{babel}
\usepackage{hyperref}



\title{Gregor Mendel}
\author{Juan M. Barrios}

\begin{document}
	
\maketitle

\begin{abstract}
Gregor Johann Mendel (20 de julio de 1822 – 6 de enero de 1884) fue un monje agustino católico y naturalista nacido en Heinzendorf, Austria (actual Hynčice, distrito Nový Jičín, República Checa) que describió, por medio de los trabajos que llevó a cabo con diferentes variedades del guisante o arveja (Pisum sativum), las hoy llamadas leyes de Mendel que rigen la herencia genética. Los primeros trabajos en genética fueron realizados por Mendel. Inicialmente realizó cruces de semillas, las cuales se particularizaron por salir de diferentes estilos y algunas de su misma forma. En sus resultados encontró caracteres como los dominantes que se caracterizan por determinar el efecto de un gen y los recesivos por no tener efecto genético (dígase, expresión) sobre un fenotipo heterocigótico.

Su trabajo no fue valorado cuando lo publicó en el año 1866. Hugo de Vries, botánico neerlandés, Carl Correns y Erich von Tschermak redescubrieron por separado las leyes de Mendel en el año 1900.\cite{Bowler2003}
\end{abstract}

\tableofcontents

\chapter{Biografía}
Gregor Mendel nació el 20 de julio de 1822 en un pueblo llamado Heinzendorf (hoy Hynčice, en el norte de Moravia, República Checa) perteneciente al Imperio austrohúngaro, y fue bautizado con el nombre de Johann Mendel. Tomó el nombre de padre Gregorio al ingresar como fraile agustino, en 1843, en el convento de agustinos de Brno (conocido en la época como Brünn). En 1847 se ordenó sacerdote.

\begin{figure}[h!]
	\centering
	\includegraphics[scale=0.8]{mendelfoto}
	\caption{Gregor Mendel}
\end{figure}

Mendel fue titular de la prelatura de la Imperial y Real Orden Austriaca del emperador Francisco José I, director emérito del Banco Hipotecario de Moravia, fundador de la Asociación Meteorológica Austriaca, miembro de la Real e Imperial Sociedad Morava y Silesia para la Mejora de la Agricultura, Ciencias Naturales y Conocimientos del País y jardinero (aprendió de su padre cómo hacer injertos y cultivar árboles frutales).

Mendel presentó sus trabajos en las reuniones de la Sociedad de Historia Natural de Brünn\cite{Mendel1901} (Brno) el 8 de febrero y el 8 de marzo de 1865, y los publicó posteriormente como Experimentos sobre hibridación de plantas (Versuche über Plflanzenhybriden) en 1866 en las actas de la Sociedad. Sus resultados fueron ignorados por completo, y tuvieron que transcurrir más de treinta años para que fueran reconocidos y entendidos.\cite{Bowler2003} Curiosamente, el mismo Charles Darwin no sabía del trabajo de Mendel, según lo que afirma Jacob Bronowski en su célebre serie/libro El ascenso del hombre.\cite{bronowski2011ascent}

Al tipificar las características fenotípicas (apariencia externa) de los guisantes las llamó «caracteres». Usó el nombre «elemento» para referirse a las entidades hereditarias separadas. Su mérito radica en darse cuenta de que en sus experimentos (variedades de guisantes) siempre ocurrían en variantes con proporciones numéricas simples.

Los «elementos» y «caracteres» han recibido posteriormente infinidad de nombres, pero hoy se conocen de forma universal con el término genes, que sugirió en 1909 el biólogo danés Wilhem Ludwig Johannsen. Para ser más exactos, las versiones diferentes de genes responsables de un fenotipo particular se llaman alelos. Los guisantes verdes y amarillos corresponden a distintos alelos del gen responsable del color.

Mendel falleció el 6 de enero de 1884 en Brünn, a causa de una nefritis crónica.

\chapter{Leyes de Mendel (1865)}

\begin{itemize}
	\item Primera ley o principio de la uniformidad: «Cuando se cruzan dos individuos de raza pura, los híbridos resultantes son todos iguales». El cruce de dos individuos homocigotas, uno de ellos dominante (AA) y el otro recesivo (aa), origina sólo individuos heterocigotas, es decir, los individuos de la primera generación filial son uniformes entre ellos (Aa).

	\item Segunda ley o principio de la segregación: «Ciertos individuos son capaces de transmitir un carácter aunque en ellos no se manifieste». El cruce de dos individuos de la F1 (Aa) dará origen a una segunda generación filial en la cual reaparece el fenotipo ``a'', a pesar de que todos los individuos de la F1 eran de fenotipo ``A''. Esto hace presumir a Mendel que el carácter ``a'' no había desaparecido, sino que sólo había sido "opacado" por el carácter ``A'' pero que, al reproducirse un individuo, cada carácter se segrega por separado.
	\item Tercera ley o principio de la combinación independiente: Hace referencia al cruce polihíbrido (monohíbrido: cuando se considera un carácter; polihíbrido: cuando se consideran dos o más caracteres). Mendel trabajó este cruce en guisantes, en los cuales las características que él observaba (color de la semilla y rugosidad de su superficie) se encontraban en cromosomas separados. De esta manera, observó que los caracteres se transmitían independientemente unos de otros. Esta ley, sin embargo, deja de cumplirse cuando existe vinculación (dos genes están muy cerca y no se separan en la meiosis).
\end{itemize}

Algunos autores obvian la primera ley de Mendel, y por tanto llaman «pri\-me\-ra ley» al principio de la segregación y «segunda ley» al principio de la transmisión independiente (para estos mismos autores, no existe una «tercera ley»).

\chapter{Experimentos de Mendel}
Mendel inició sus experimentos eligiendo dos plantas de guisantes que diferían en un carácter, cruzó una variedad de planta que producía semillas amarillas con otra que producía semillas verdes; estas plantas forman la llamada generación parental (P).

Como resultado de este cruce se produjeron plantas que producían nada más que semillas amarillas, repitió los cruces con otras plantas de guisante que diferían en otros caracteres y el resultado era el mismo, se producía un carácter de los dos en la generación filial. Al carácter que aparecía lo llamó carácter dominante y al que no, carácter recesivo. En este caso, el color amarillo es uno de los caracteres dominantes, mientras que el color verde es uno de los caracteres recesivos.

Las plantas obtenidas de la generación parental se denominan en conjunto primera generación filial (F1).

Mendel dejó que se autofecundaran las plantas de la primera generación filial y obtuvo la llamada segunda generación filial (F2), compuesta por plantas que producían semillas amarillas y por plantas que producían semillas verdes en una proporción 3:1 (3 de semillas amarillas y 1 de semillas verdes). Repitió el experimento con otros caracteres diferenciados y obtuvo resultados similares en una proporción 3:1.

A partir de esta experiencia, formuló las dos primeras leyes.

Más adelante decidió comprobar si estas leyes funcionaban en plantas diferenciadas en dos o más caracteres, para lo cual eligió como generación parental a plantas de semillas amarillas y lisas y a plantas de semillas verdes y rugosas.

Las cruzó y obtuvo la primera generación filial, compuesta por plantas de semillas amarillas y lisas, con lo cual la primera ley se cumplía; en la F1 aparecían los caracteres dominantes (amarillos y lisos) y no los recesivos (verdes y rugosos).

Obtuvo la segunda generación filial autofecundando a la primera generación filial y obtuvo semillas de todos los estilos posibles, plantas que producían semillas amarillas y lisas, amarillas y rugosas, verdes y lisas y verdes y rugosas; las contó y probó con otras variedades y se obtenían en una proporción 9:3:3:1 (9 plantas de semillas amarillas y lisas, 3 de semillas amarillas y rugosas, 3 de semillas verdes y lisas y una planta de semillas verdes y rugosas).

\chapter{Mendel y la apicultura}
Un aspecto no muy conocido fue su dedicación durante los últimos 10 años de su vida a la apicultura. Mendel reconoce que las abejas resultaron un modelo de investigación frustrante. Es probable que el experimento realizado con abejas tuviera como objetivo confirmar la teoría de la herencia.

En 1854 Mendel discute en Silesia con los apicultores la hipótesis de Jan Dzierzon que enuncia que las reinas infértiles o los huevos que no son fecundados por esperma de los machos producen zánganos, produciéndose reproducción sexual en las hembras y reproducción asexual en los machos o zánganos. A este proceso Jan Dzierzon lo denominó partenogénesis.

La teoría de Dzierzon fue confirmada por hibridación, si bien el cruce de abejas es difícil, pues durante el vuelo nupcial de la reina no debe haber zánganos extraños. Por ello, Mendel construyó una jaula de tejido de cuatro metros de largo y cuatro de alto, situando la colmena en el exterior de ella, para lograr el objetivo deseado que era realizar los cruces necesarios para lograr los híbridos de diferentes razas de abejas. Pero la teoría de Dzierzon no se confirmó en vida de Mendel. Seguramente lo que Mendel pretendía era probar la segregación de caracteres genéticos.

El director de la Sociedad de Apicultura de Brünn (Brno), Ziwansky, proveyó diferentes razas de abejas de la especie Apis mellifera: italianas (Apis mellifera ligustica), carniolas (Apis mellifera carnica), egipcias y chipriotas, que los apicultores locales reproducían. Las chipriotas fueron obtenidas directamente de Chipre por el conde Kolowrat. Algunas de las abejas con diferencias de colores fueron obtenidas de Pernambuco (estado) (Brasil), incluidos algunos especímenes de Sudamérica. Estos fueron enviados por el profesor Macowsky a Mendel y eran abejas de la especie Trigona lineata, melipónidos o abejas sin aguijón, criadas durante dos años sucesivos.

Mendel fue un activo miembro de la Sociedad de Apicultura de Brünn (Brno) y en 1871 fue nombrado presidente de la misma. Entre el 12 y el 14 de septiembre de 1871, Mendel y Ziwansky fueron delegados por la Asociación de Apicultura de Brünn (Brno) al Congreso de Apicultura en lengua germana a desarrollarse en Kiel. En 1873 Mendel declinó la presidencia y en 1874 fue reelecto, pero por circunstancias personales privadas indicó que le resultaba imposible ocupar el cargo. En 1877 se afirma, en Honigbienen (la revista de la Asociación), que el prelado de las abejas poseía 36 colmenas. Pero en realidad el interés biológico de Mendel residía en la relación que tienen las abejas con las flores.

\chapter{Honores}

\begin{itemize}
	\item En Sevilla, España, hay una calle con su nombre.
	\item En 1994, la Universidad Mendel pasa a llamarse con su nombre en su honor.
	\item Un colegio mayor de Madrid lleva su nombre.
	\item En Argentina, el 8 de febrero se celebra el día del Genetista, en honor a la primera fecha de la presentación de sus trabajos.
\end{itemize}

\chapter{Epónimos}
\begin{itemize}
	\item (Amaryllidaceae) Amaryllis mendelii Hort.
	\item (Aspleniaceae) Asplenium  mendelianum D.E.Mey.
	\item (Asteraceae) Hieracium  mendelii Peter
	\item (Cactaceae) Mammillaria mendeliana (Bravo) Werderm.
	\item (Cactaceae) Neomammillaria mendeliana Bravo
	\item (Chenopodiaceae) Chenopodium  mendelii F.Dvořák
	\item (Orchidaceae) Aerides mendelii Hort. ex E.Morren
	\item (Orchidaceae) Cattleya mendelii L.Linden \& Rodigas	
\end{itemize}

\chapter{Abreviatura}
La abreviatura Mendel se emplea para indicar a Gregor Mendel como autoridad en la descripción y clasificación científica de los vegetales. (Véase listado de todos los géneros y especies descritos por este autor en IPNI).

\bibliographystyle{plain}
\bibliography{mendelbibliografia}
\addcontentsline{toc}{chapter}{Bibliografía}

\end{document}