\documentclass[letterpaper,12pt]{book}

\usepackage[utf8]{inputenc} % Soporte para acentos
\usepackage[T1]{fontenc}    
\usepackage[spanish,es-tabla]{babel} % Español
\usepackage{amsmath}		% Soporte de símbolos adicionales (matemáticas)
\usepackage{amssymb}		
\usepackage{amsfonts}
\usepackage{latexsym}
\usepackage{graphicx}
\usepackage{tabularx}
\usepackage{cite}
\usepackage{hyperref}
\usepackage{epstopdf}%Esta instruccion nos permte insertar imagenes y generar un pdf

% Información para el título
\title{Notas al Pie \\ 
	\vspace{1cm}Tablas de Contenidos \\ 
	\vspace{1cm}Índice de Figuras y Tablas}
\author{J. Luis Torres}

% Indicamos una separación entre los párrafos
\parskip=3mm

% Eliminamos la sangría de los párrafos
\parindent=0mm

\begin{document}

\maketitle
\frontmatter %incluir partes como prologos prefacios

\tableofcontents

\listoffigures
\addcontentsline{toc}{chapter}{Lista de figuras}

\listoftables



\chapter{Prólogo}

En el futuro incluiremos un prólogo en esta sección.

\mainmatter % a partir de aquí ya empieza la numeración 

\chapter{Gráficas sencillas con Maple 15}

%\thispagestyle{empty} %quitar la numeracion de la pagina

\emph{Maple} es considerado un \emph{Sistema de Álgebra Computacional}, proporciona múltiples
funcionalidades al usuario entre las que se pueden listar las siguientes:

\begin{itemize}
	\item Genera expresiones en simbología matemática.
	\item Puede desplegar gráficas de diversos tipos de funciones en 2 y 3 dimensiones.
	\item Genera animaciones en dos y tres dimensiones de diferentes tipos de funciones.
	\item Puede generar código en diversos lenguajes de programación.
	\item Genera código en \LaTeX{}.
	\item Maneja números de precisión infinita \footnote{En realidad si existen límites pero están 
		determinados por la capacidad de la plataforma, no del sistema}.
	\item Puede exportar gráficas y animaciones a diferentes formatos.
\end{itemize}

Veamos un ejemplo de creación de gráficas sencillas.

Las siguientes instrucciones nos permiten definir una función y generar su gráfica 
em Maple 15:

\begin{verbatim}
k:= x -> sin(x) + exp(cos(exp(x)));

plot(k(x), x=-10..4.5);
\end{verbatim}

Estas instrucciones nos permiten generar la gráfica que se muestra en la figura \ref{cap1f1}.

\begin{figure}[h!]
\centering
\includegraphics[scale=0.5]{graficasMapleplot2d1.eps}
\caption{Gráfica de $sen \left( x \right) +{{\rm e}^{\cos \left( {{\rm e}^{x}} \right) }}$}\label{cap1f1}
\end{figure}

Veamos otro ejemplo:

Las siguientes instrucciones nos permiten generar la gráfica de la función 
$\displaystyle {x}^{4}\sin \left( {x}^{3} \right) -{x}^{3}\cos \left( {x}^{2}
 \right) +{x}^{2}\sin \left( x \right) -x$

\begin{verbatim}
plot(x^4*sin(x^3)-x^3*cos(x^2)+x^2*sin(x)-x, x = -1.8 .. 1.8);
\end{verbatim}

La gráfica se muestra en la figura \ref{cap1f2}.

\begin{figure}[h!]
\centering
\includegraphics[scale=0.5]{graficasMapleplot2d2.eps}
\caption{Gráfica de ${x}^{4}\sin \left( {x}^{3} \right) -{x}^{3}\cos \left( {x}^{2}
 \right) +{x}^{2}\sin \left( x \right) -x$}\label{cap1f2}
\end{figure}

En la tabla \ref{tab:cap1t1} podemos consultar algunas instrucciones de \emph{Maple} que nos
permiten generar gráficas en dos dimensiones.

\begin{table}[h!]
	\caption{Instrucciones de Maple para gráficas en 2D}
	\begin{center}
		\begin{tabular}{|c|c|} \hline
			Instrucción & Tipo de gráfica generada \\ \hline
			plot() & Gráficas en 2D de funciones explicitas \\ \hline
			plot() & Gráficas en 2D de funciones paramétricas \\ \hline
			polarplot() & Gráficas en coordenadas polares \\ \hline
			implicitplot() & Gráficas implícitas en 2D \\ \hline
			complexplot() & Gráficas de expresiones complejas \\ \hline
			contourplot & Gráficas de contornos \\\hline
		\end{tabular}
	\end{center}
	\label{tab:cap1t1}
\end{table}

\chapter{Gráficas de Funciones con Discontinuidades}

\thispagestyle{empty}

De manera predeterminada, cuando hacemos uso de funciones para la creación de gráficas en 
dos dimensiones en \emph{Maple}, si los argumentos corresponden a expresiones con 
discontinuidades en el intervalo indicado, este sistema desplegará lineas verticales en
los intervalos de las discontinuidades.

Para poder generar gráficas de funciones de este tipo, la opción
\emph{discont=true} nos permite indicar a \emph{Maple} que éstas deben eliminarse de la
gráfica. La forma en la que incluimos esta opción es la siguiente:

\begin{verbatim}
plot(funcion(x), x = intervalo, {rango}, discont = true);
\end{verbatim}

El intervalo para $x$ debe indicarse en la forma $x=limInferior..limSuperior$, por ejemplo:
$x=-\pi..\pi$. También se puede incluir un tercer argumento para indicar el rango en el que
se debe desplegar la gráfica, en algunos casos es útil porque permite restringir el despliegue,
en caso de que el rango en el intervalo para $x$ sea muy grande.

Por ejemplo, la siguiente instrucción nos permite desplegar una gráfica de la función \emph{tan(x)}:

\begin{verbatim}
plot(tan(x), x = -10 .. 10, -50 .. 50, discont = true);
\end{verbatim}

La gráfica se puede ver en la figura de abajo \ref{cap2f1}.

\begin{figure}[h!]
\centering
\includegraphics[scale=0.5]{graficasMapleplot2d4.eps}
\caption{Texto de figura 4}\label{cap2f1}
\end{figure}

\chapter{Gráficas con Múltiples Funciones}

\thispagestyle{empty}

La siguiente instrucción nos permite incluir las gráficas de varias funciones en un mismo despliegue:

\begin{verbatim}
plot({x, cos(x), sin(x)}, x, color = [blue, brown, green]);
\end{verbatim}

La gráfica se puede observar en la figura siguiente \ref{cap3f1}.

% Insertar la imagen graficasMapleplot2d5
\begin{figure}[h!]
\centering
\includegraphics[scale=0.5]{graficasMapleplot2d5.eps}
\caption{Texto de figura 5}\label{cap3f1}
\end{figure}

\chapter{Gráficas de Contornos}

La siguiente instrucción nos permite desplegar una gráfica de contornos para una función de
$x$ y $y$, en el rectángulo indicado.

\begin{verbatim}
contourplot(sin(x)*y, x = -5 .. 5, y = -4 .. 4)
\end{verbatim}

Podemos ver la gráfica en la figura de abajo \ref{cap4f1}.

\begin{figure}[h!]
\centering
\includegraphics[scale=0.35]{graficasMapleplot2d6.eps}
\caption{Texto figura 6}\label{cap4f1}
\end{figure}


\chapter{Gráficas en Coordenadas Polares}

\emph{Maple} está estructurado en forma modular, de tal manera que muchas de las
funciones, constantes, símbolos y otras características, están contenidas en forma
de archivos o conjuntos de archivos conocidos como \emph{paquetes}. 

Para poder hacer uso de funciones contenidas en paquetes de \emph{Maple}, es 
necesario cargar éstos haciendo uso de la siguiente instrucción:

\begin{verbatim}
with(paquete);
\end{verbatim}

Por ejemplo, para crear gráficas de funciones en coordenadas polares, es necesario
cargar previamente el paquete \emph{plots} de la siguiente forma:

\begin{verbatim}
with(plots);
\end{verbatim}

A continuación ya podemos hacer uso de las funciones contenidas en el paquete.

Una de las funciones contenidas en \emph{plots} es \emph{polarplot}, la cual nos
permite crear gráficas de funciones en coordenadas polares.

Por ejemplo, la siguiente instrucción generará una gráfica para 
$[t, sen()t]$ en el intervalo $\{0, 4\pi\}$, en coordenadas polares:

\begin{verbatim}
polarplot([t, sin(t), t = 0 .. 4*Pi], color = blue);
\end{verbatim}

La gráfica se puede apreciar en la figura de abajo \ref{cap5f1}.

% Insertar graficasMapleplot2d7
\begin{figure}[h!]
\centering
\includegraphics[scale=0.5]{graficasMapleplot2d7.eps}
\caption{Texto de figura 5}\label{cap5f1}
\end{figure}


\chapter{Gráficas en 3D}

La siguiente instrucción nos muestra una forma de generar gráficas de funciones en tres dimensiones en \emph{Maple}:

\begin{verbatim}
plot3d(x^2*cos(y), x = -5 .. 5, y = -5 .. 5)
\end{verbatim}

La gráfica se puede apreciar en la figura \ref{cap6f1}.

\begin{figure}[h!]
\centering
\includegraphics[scale=0.35]{graficasMapleplot2d10.eps}
\caption{Gráfica de $x^2\cdot cos(y)$}\label{cap6f1}
\end{figure}

En la instrucción anterior solamente incluimos como argumentos de \emph{plot3d} la función 
a gráficar y los intervalos para las variables.

La función \emph{plot3d} soporta un conjunto de opciones, por ejemplo, la siguiente instrucción 
nos permite crear la misma gráfica mostrada en la figura \ref{cap6f1} pero con una estructura de
alambres.

\begin{verbatim}
plot3d(x^2*cos(y), x=-5..5, y=-5..5, style=WIREFRAME);
\end{verbatim}

Podemos ver la gráfica resultante en la figura siguiente.

% Insertar graficasMapleplot2d11.eps

\chapter{Tablas}

\section{Tabla de verdad}

En la tabla \ref{tab:cap8t1} podemos ver la tabla de verdad $p \rightarrow q$:

\begin{table}[ht]
	\caption{Tabla de verdad $p \rightarrow q$}
	\begin{center}
		\begin{tabular}{|c|c|c|} \hline
			$p$ & $q$ & $p \rightarrow q$\\\hline
			0 & 0 & 1 \\
			0 & 1 & 1 \\
			1 & 0 & 0 \\
			1 & 1 & 1 \\\hline
		\end{tabular}
	\end{center}
	\label{tab:cap8t1}
\end{table}

\section{Algunos símbolos útiles}

La tabla de abajo nos muestra algunos símbolos matemáticos \ref{tab:cap7t2}

\begin{table}[h!]
	\caption{Instrucciones de Maple para gráficas en 2D}
	\begin{center}
		\begin{tabular}{|c|c|} \hline
			Símbolo & Comando \\ \hline
			$$ \mathbb{R}$$ &  \verb@\mathbb{R}@  \\ \hline
			$$\mathbb{Z}$$ & \verb@\mathbb{Z}@ \\ \hline
			$$\mathbb{Q}$$ & \verb@\mathbb{Q}@ \\ \hline
		\end{tabular}
	\end{center}
	\label{tab:cap7t2}
\end{table}

% Insertar lo siguiente en una tabla
%			Símbolo & Comando \\ \hline
%			$$\mathbb{R}$$ & \verb@\mathbb{R}@	\\ \hlinehttps://accounts.google.com/ServiceLogin?service=chromiumsync&sarp=1&rm=hide&continue=https%3A%2F%2Fwww.google.com%2Fintl%2Fes-419%2Fchrome%2Fblank.html%3Fsource
%			$$\mathbb{Z}$$ & \verb@\mathbb{Z}@ \\ \hline
%			$$\mathbb{Q}$$ & \verb@\mathbb{Q}@
%		\end{tabular}


\appendix
\chapter{Bibliográfia con \LaTeX{}}
\addcontentsline{toc}{chapter}{Apéndice A. Bibliográfia con \LaTeX{}}

\section{El entorno \textit{thebibliography}}

Existen varias formas de incluir citas bibliográficas en un 
documento de \LaTeX{}. Una de ellas consiste en incluir las
citas al final del documento, en una sección como la siguiente:

\begin{verbatim}
\begin{thebibliography}{99}

\bibitem{spivakCalc} Spivak, Michael; Calculus; Reverte; 1996.

\bibitem{Lamport} Lamport, L.; \LaTeX{}; Addison-Wesley. 1996.

\end{thebibliography}
\end{verbatim}

Una vez incluida esta bibliografía, podemos hacer referencia a
una de las entradas mediante la siguiente expresión:

\begin{verbatim}
... el teorema del valor medio se puede consultar en
 \cite{spivakCalc} ... y \cite{Lamport} incluye un capitulo sobre
 como insertar gráficas en \LaTeX{} ...
\end{verbatim}

\section{Citas bibliográficas con Bib\TeX{}}

Otra forma más recomendable de incluir citas bibliográficas es
mediante el uso de Bib\TeX{}. En este caso es necesario crear
una ``\textit{base de datos}'' en un archivo de texto con 
extensión \textbf{.bib}, haciendo uso de la siguiente estructura:

\begin{verbatim}
@tipoDeCita{Llave,
	propiedad1="valor1",
	propiedad2="valor2",
	...
}
\end{verbatim}

En esta estructura, la \emph{Llave} es la expresión con la cual 
se hará referencia a una cita y en las propiedades se incluyen
los diferentes datos de ésta, tales como autor, año, páginas,
capítulo, abstract, entre otras \footnote{En la sección A.3
se incluyen algunas de las propiedades permitidas.}.

El \emph{tipoDeCita} indica en que clase de documento se incluirá
\footnote{La sección A.4 incluye algunos de los tipos
aceptados.}.

Este archivo debe estar ubicado en el mismo directorio en el que
se encuentra nuestro archivo \textbf{.tex}.

Una vez creada la base de datos, ésta puede ser incluida en el
documento mediante las siguientes instrucciones:

\begin{verbatim}
\bibliographystyle{Estilo}
\bibliography{basededatos1[,basededatos2,...]}
\end{verbatim}

Donde \emph{Estilo} indica el formato en el que se incluirán las
citas de la bibliografía y \emph{basededatosX} indica el archivo
o archivos de los cuales se tomarán los datos de las citas.

Algunos de los estilos más usados son: \emph{plain},
\emph{apalike}, \emph{alpha}, \emph{abbrv} y \emph{unsrt}.

Por ejemplo, las citas incluidas al inicio del apéndice se 
pueden incluir en un archivo \textbf{mibiblio.bib}, en el
siguiente formato:

\begin{verbatim}
@book{spivakCalc,
author="Spivak, Michael",
title="Calculus",
editor="Reverte",
year="1996",
pages="944"
}

@book{Lamport,
author="Leslie Lamport",
title="\LaTeX",
editor="Addison-Wesley",
year="1996"
}
\end{verbatim}

A continuación incluimos las siguientes instrucciones para 
insertar las citas:

\begin{verbatim}
\bibliographystyle{amsplain}
\bibliography{mibiblio.bib}
\end{verbatim}

En este caso hacemos uso del estilo de la \emph{American 
Mathematical Society}. Con este estilo las citas se ordenan
alfabéticamente y se colocan etiquetas numéricas a cada una
de ellas.

Una de las ventajas de manejar la bibliografía de esta manera es
que solamente se incluiran las entradas citadas en el documento.

\newpage 

\section{Bib\TeX{}, propiedades soportadas}

Algunas de las propiedades soportadas por Bib\TeX{} son:

\begin{tabular}{llll}
address & abstract & author & booktitle \\
chapter & contents & copyright & crossref \\
edition & editor & howpublished & institution \\
ISBN & ISSN & journal & key \\
keywords & language & month & note \\
number & organization & pages & publisher \\
school & series & title &url \\
volume & year & & 
\end{tabular}

\section{Tipos de citas}

Algunos de los tipos de citas válidos en los archivos de 
\emph{bases de datos} de Bib\TeX{} son:

\begin{tabular}{lll}
article & book & booklet \\
conference & inbook & incollection \\
inproceedings & manual & mastersthesis \\
misc & other & phdthesis \\
proceedings & techreport & unpublished
\end{tabular}


Para conocer el TVM consulte \cite{spivakCalc}. %se debe compilar en modo Bibtex y no pdflatex dos veces
%sirve para insertar la bibliografía y va al final del documento, se debe de incluir el archivo.bib
%donde vienen definidas las etiquetas para cada bibliografía
\bibliographystyle{amsplain}
\bibliography{mibiblio.bib}
\addcontentsline{toc}{chapter}{Bibliografia}

\end{document}



